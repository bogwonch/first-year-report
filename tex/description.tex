\documentclass[report.tex]{subfiles}
\begin{document}

\section{Project description}

The aim of the thesis is to show how authorization logics can be used to make
security decisions in mobile devices.  Currently security decisions are made
manually by smart phone users and it is our belief that by automating these
choices users can avoid having to make security decisions and their overall
security be improved.  To do this we plan to look at the following areas: 

\begin{itemize}

  \item \emph{To instantiate a logic of authorization that allows us to model
      the trust relationships between the components of an operating system and
      the users.}  This will include analysis tools as principals and allow
    making decisions based on signed statements from them.  The logic must be
    able to model what happens when apps can collude.  The logic may be based
    off of earlier work on the \emph{SecPAL~language}\cite{Becker:2006vh} that
    was used for distributed access control decisions.

  \item \emph{To explore how security policies change with time and when apps
      can collude.}  A user's security policy need not be static.  People change
    jobs and may bring old devices to new environments requiring new security
    policies.  Apps can collude: two apps might meet a security policy when
    considered on their own but together they might act to share data
    inappropriately.  Over time an app might want greater access and increased
    permissions to support new functionality.  It is not obvious how to write
    and check security policies written in SecPAL for these scenarios and how
    to enforce the policy at runtime.

  \item \emph{To implement an app store that serves users only the apps that
      meet their security policies.}  This will include a user-study where we
    evaluate how well users comprehend their policies and the decisions made for
    them. This may lead into generating proof-carrying code certificates for
    apps that allow a device to check that their policy was met without having
    to do the full inference themselves.

  \item \emph{To model the decisions and trust relationships inherent in Android
      and other mobile operating systems.}  This will allow us to write a
    security policy that describes the current state in these systems and serve
    as a base to compare other systems against. 

  \item \emph{To study how users understand their security policies and the
      ways these policies are enforced.}  One of the advantages of SecPAL is
    that it is more readable compared to other authorization logics and access
    control languages.  Whilst end-users might not want to write their own
    policies they should be able to comprehend what a policy means, and they
    should be able to understand why their policy allows some decisions and not
    others.

\end{itemize}

\subsection{A Logic of Authorization For Mobile Devices}


\subsection{Compositional Policies Over Time}

Consider the case where a user has a smart phone and they are buying apps.  The
user must decide if they want to install an app: to do this they apply a series
of judgements called their \emph{security policy}.  

Whilst the user has their own security policy they apply they also have other
security policies they implicitly follow.  If they are downloading apps from an
app store they are also subject to the security policy of the app store and what
it is willing to sell.  If the phone runs in a corporate environment then they
may also be subject to the company's corporate policy.  Finally the operating
system itself may have certain restrictions on what it will allow: for example
the APK app format used on Android can also be installed on Blackberry, and Mer
operating systems.  Each of these systems may add additional restrictions that
may make some apps not installable.  An example of how this kind of policy might
be written is shown in Figure~\ref{example:composition}.

\begin{marginfigure}\label{example:composition}
  \begin{lstlisting}[language=SecPAL]
Phone says app is-installable
  if app meets UserSecurityPolicy,
     app meets AppStorePolicy,
     app meets ITDeptPolicy,
     app meets OSPolicy.

Phone says User can-say inf
  app meets UserSecurityPolicy.

Phone says PlayStore can-say 0
  app meets AppStorePolicy.

Phone says ITAdmin can-say inf
  app meets ITDeptPolicy.
  \end{lstlisting}
  \caption{A compositional security policy where an installation policy for a
    phone is dependent on other security policies.}
\end{marginfigure}
     
The phone might use this policy for a while, but then the user changes jobs.
Now they have to meet a new \code{ITDeptPolicy} set by a different administrator.
Should any installed apps be uninstalled if they don't meet the new policy?  If
we already have a certificate showing the apps passed the old policy can we
reuse it to create a new certificate that shows the app meets any additional
restrictions?

Whilst other authorization logics have looked at making one-time decisions about
whether to allow a computer to make a decision; there has been less work on
modelling these policies over time and seeing how a changing security policy
affects a changing device.  This could add novelty.

Alternatively say there is an app which the developer is continually improving
and adding new features.  When the app is installed it may meet the security
policy but with increasing features requiring access to more permissions and
introducing more complexity or a change of advert library the app no longer
meets the security policy.

Should the app be removed?  If the app is used every day by then the user may
not be pleased that the phone has decided to break their favorite
app\footnote{Though anecdotal evidence would suggest users tend to blame apps
  for failing rather than the frameworks they apps rely on; regardless of who is
  really to blame.}; equally just stopping updates for the app increases app
version fragmentation and reduces security by rejecting bug fixes.  Allowing the
update isn't correct either as it allows a means to break the security policy.

Whilst there have been several papers looking at (and proposing methods to stop)
excessive permissions in applications\cite{Felt:2011kj}\cite{Vidas:2011wr} there
has not been a thorough review of how permissions change for apps over time
and between versions of the same app. 

% TODO ADD BIT EXPLAINING HOW TO DO THIS


\subsection{Personally Curated App Stores}

The primary method of software distribution on mobile devices is through an app
store.  On iOS users have the \emph{App Store}: a curated market place run by
Apple (though other, albeit clunkier, distribution mechanisms do exist for even
non-jailbroken phones) that is perceived as being picky about the apps it sells.

On Android users have a far greater choice of marketplace.  The \emph{Play
  Store} is the standard app store distributed by Google and is less moderated
than Apple's store.  Amazon have their own app store that serves as a more
curated version of Google's offering and the default on their Kindle tablets.
Other app stores target specific regions: such as \emph{gFan} in China, and the
\emph{SK~T-Store} in Korea.  Some, such as \emph{Yandex.Store, AppsLib and
  SlideMe}, are pre-installed by OEMS who can't or don't want to meet Google's
requirements for the PlayStore.  The \emph{F-Droid} store only delivers open
source apps. Others exist to distribute pirated apps.  On average eight
percent\cite{AQUILINO:2013wr} of the apps in each of these alternative market
places is malware. The Play Store contains very little malware however (0.1\% of
total apps), whilst a third of the app in the Android159 store were found to be
malicious.

Each of these app stores have a different security policy.  They enforce these
policies when they pick which apps to sell to their users.  By using an
authorization logic to decide whether apps will meet a security policy we have
the ability to create a new kind of app store where offerings are tailored to
the user's security policy.  By creating app stores tailored to a security
policy we also give ourselves a way to empirically measure how restrictive a
security policy is: we can measure the number of apps offered inside the stores.

To enhance the trust in the store by the user digital evidence could be offered
with the app which would give devices a practical means to check that the app is
supported by their security policy without having to re-run all the checks
themselves; this should also save device battery life.  Proof-carrying
authentication\cite{Appel:1999dq} and authorization logics such as
BLF\cite{Whitehead:2004bu} have already introduced ideas from proof-carrying
code into authorization logics. The focus of this work, however, has been on
access control where a user is providing a proof that they have the credentials
to access a resource.   In the scenario we propose the role of the user is
reversed: the store offers many proofs to the user to increase their trust in
its wares; rather than the user offering one specific proof to prove they have
the right to complete a certain action.



\end{document}

