\documentclass[report.tex]{subfiles}
\begin{document}

\section{Introduction}

When an app is installed on Android it prompts the user to accept a list of
privileges granted to the app.  The user makes the decision based on what they
know about the app and their own security policies.  In practice a large number
of users accept the permissions whatever.  This is problematic because some apps
are over privileged\cite{Felt:2011kj} and some are malicious\cite{Zhou:2012cf}.
Other apps are considered to be \ac{PUS} because though they are not malicious
in themselves they handle data in a way that is not in the user's interests.

More generally users and computers make decisions, whether it update an app;
whether to connect to a website, based on security policies and trust
relationships.  These security policies may include the use of tools or experts
to decide whether something is malicious.  For instance a user may trust a
firewall program to enforce their network policy; and they may trust a tool like
\emph{Shorewall} to generate the actual policy for them.  Alternately a user
might wish to be able to install apps but only trust apps \emph{Amazon} have
vetted to be installed on their device.  \emph{The aim of this project is to
  formalize these security policies so they can be studied and enforced
  automatically.}

Mobile operating systems are similar to existing systems but come with a
different trust model and are used in a different manner.  Software is
downloaded from app stores, Apps run within sandboxes and must collaborate and
collude to share data with other apps. These devices contain more personal data
than ever before: sensors tracking users' locations,  gyroscopes measuring how
users move, and microphones listening to users calls.  The \ac{BYOD} movement
empowers users to take the devices they have at home and bring them into work.
This creates a tension between how the corporate IT department may require
employees to use their devices and the user's policies on how they want to use
their devices.  These features add a novel challenge to modelling these devices
and the stores and users surrounding them.  

Formalizing the policies allows comparisons to be made between different systems
and users.  When comparisons are made between the two biggest mobile OSs,
iOS~and~Android, the comparisons is informal: we say iOS is more closed, more of
a \emph{walled garden}, Android is more permissive.  With a formal language to
describe system policy we can make a precise comparison.  It allows us to 

There is decades of research on analysis tools.  These tools can infer complex
security properties about the code and systems they analyze.  What there is
missing is a glue layer between the assurances these tools can give and the
policies users are trying to enforce.  By using an \emph{authorization logic} as
the glue layer we can enforce the policy by building on the work on access
control in distributed systems.  Our static analysis tools can be trusted to
give statements about code, with other principles, that can be combined to
implement a security policy.




\end{document}

